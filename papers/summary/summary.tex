%        File: summary.tex
%     Created: Thu Jul 12 07:00 PM 2012 H
% Last Change: Thu Jul 12 07:00 PM 2012 H
%
\documentclass[a4paper]{scrartcl}
\usepackage{hyperref}
\usepackage{bibentry}
\nobibliography*
\begin{document}
\title{Summaries of Papers on Opinion Formation}
\author{Mathis Antony}
\maketitle
\tableofcontents
\section{Who's Talking First? Consensus or Lack Thereof in Coevolving Opinion
    Formation Models}
  \subsection{Reference}
    \bibentry{PhysRevLett.100.158701}
  \subsection{Model}
    \begin{itemize}
      \item Voter-like Model (VM), direct \ldots (dVM): selected agent adopts,
        reverse \ldots (rVM): selected agent teaches
      \item Rewiring probability $\Phi$ if opinion of neighbour differs
      \item Opinion update with probability $1-\Phi$  
      \item Naming Game (NG), with intermediate 0 state: dNG, rNG
    \end{itemize}
  \subsection{Results}
    \begin{itemize}
      \item $t_c$: time needed to reach consensus
      \item $t_c \propto N^\alpha$ for static networks
      \item $t_c \propto \ln N$ dVM (coevolving)
      \item $t_c \propto \exp N$ rVM (coevloving)
      \item $m=n_+-n_-$ (magnetization) MF approximation shows positive
        feedback for dVM ($m_\textrm{stable}=\pm1$), negative for rVM
        ($\ldots=0$) 
      \item $t_c \propto \ln N$ for NG on homogeneous networks
      \item Surface tension introduced by 0 states. Always generate positive
          feedback for change in $m$. Consensus reached by minimization of
          interface instead of finite size fluctuations in the VM.
    \end{itemize}

\section{Opinion Formation on Adaptive Networks with Intensive Average Degree}
  \subsection{Reference}
    \bibentry{PhysRevE.79.046104}
  \subsection{Model}
    \begin{itemize}
      \item 2 opinions ($\sigma_j\pm1$).
      \item Update randomly chosen spin by majority rule. I.e. Glauber Dynamics
        with 0 noise.
      \item Reassing all links with probability $\tilde p$ ($\tilde q$) where
        $\tilde p$ and $\tilde q$ are of order $1/N$.
      \item Due to the ``strangeness'' of the model. A simple birth-death
        process occurs and equations for birth and death rate can be written
        down.
    \end{itemize}
\bibentry{RevModPhys.81.591}
\bibliographystyle{unsrt}
\bibliography{../refs/refs.bib}
\end{document}


